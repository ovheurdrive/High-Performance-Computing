%----------------------------------------------------------------------------------------
% PACKAGES AND OTHER DOCUMENT CONFIGURATIONS
%----------------------------------------------------------------------------------------

\documentclass{article}

\usepackage{fancyhdr} % Required for custom headers
\usepackage{lastpage} % Required to determine the last page for the footer
\usepackage{extramarks} % Required for headers and footers
\usepackage[usenames,dvipsnames]{color} % Required for custom colors
\usepackage{graphicx} % Required to insert images
\usepackage{courier} % Required for the courier font
\usepackage{lipsum} % Used for inserting dummy 'Lorem ipsum' text into the template
\usepackage{framed}
\usepackage[utf8]{inputenc}
\usepackage{stmaryrd}
\usepackage{algpseudocode}
\usepackage{amsthm}
\usepackage{amsfonts}
% Margins
\topmargin=-0.45in
\evensidemargin=0in
\oddsidemargin=0in
\textwidth=6.5in
\textheight=9.0in
\headsep=0.25in

\linespread{1.1} % Line spacing

% Set up the header and footer
\pagestyle{fancy}
\lhead{\hmwkAuthorName} % Top left header
\chead{\hmwkTitle} % Top center head
\rhead{\firstxmark} % Top right header
\lfoot{\lastxmark} % Bottom left footer
\cfoot{} % Bottom center footer
\rfoot{Page\ \thepage\ sur\ \protect\pageref{LastPage}} % Bottom right footer
\renewcommand\headrulewidth{0.4pt} % Size of the header rule
\renewcommand\footrulewidth{0.4pt} % Size of the footer rule

\setlength\parindent{0pt} % Removes all indentation from paragraphs

%----------------------------------------------------------------------------------------
% NAME AND CLASS SECTION
%----------------------------------------------------------------------------------------

\newcommand{\hmwkTitle}{Dépendances et parallélisation} % Assignment title
\newcommand{\hmwkClass}{High Performance Computing} % Course/class
\newcommand{\hmwkAuthorName}{Plessia Stanislas} % Your name
%----------------------------------------------------------------------------------------
% TITLE PAGE
%----------------------------------------------------------------------------------------

\title{
\LARGE{\textbf{\hmwkClass}}\\
\vspace{0.5in}
\large{\textbf{\hmwkTitle}}
\vspace{3in}
}

\author{\textbf{\hmwkAuthorName}}
\date{Février 2018} % Insert date here if you want it to appear below your name

\newtheorem*{theorem}{Théorème}
%----------------------------------------------------------------------------------------

\begin{document}

\maketitle

\newpage

\subsection*{Parallélisation d'une boucle}


\begin{theorem}
Une boucle de programme est parallélisable si et seulement si deux instructions ou séries d'instructions associées à des itérations différentes ne présentent aucunes dépendances de données ou de sorties.
\end{theorem}

\begin{proof}[Preuve]
Soit $B$ une boucle de programme.\\
Pour chaque itération de $B$ d'indice $i \in \llbracket 1,n \rrbracket$, nous notons $B_i$ l'ensemble des instructions de cette itération.\\

Nous cherchons à montrer que $\forall (i,j) \in \llbracket 1,n \rrbracket^2$, $B_i$ et $B_j$ sont permutables (ie le programme est parallélisable) si et seulement si $B_i$ et $B_j$ n'ont aucune dépendances de données ou de sorties.\\

$\Rightarrow$ Supposons $B$ parallélisable, i.e. sa sortie ne dépend pas de l'ordre des $B_i \forall i$.\\

Supposons qu'il existe une dépendance de données: $\exists (i,j), In(B_i) \cap Out(B_j) \neq \emptyset$.\\

Dans ce cas, commme $Out(B_i) \gets In(B_i)$, $Out(B_i)$ dépend de $Out(B_j)$
La sortie de $B_i$ dépend de l'exécution ou non de $B_j$ : on arrive à une contradiction.\\

Supposons maintenant qu'il existe une dépendance de sorties. $\exists (i,j), Out(B_i) \cap Out(B_j) \neq \emptyset$.
On reviens au cas précédent ou la sortie de $B_i$ va dépendre de l'éxecution ou non de $B_j$ : on a de nouveau une contradiction.\\

$\Leftarrow$ Supposons qu'il n'y ait aucune dépendance dans $B$.\\
Soit $s$ la sortie d'une instruction : $\exists! k, s \in Out(B_k)$.\\

Par indépendance de sorties, $k$ est unique.\\
On sait donc que $s$ ne dépend que de $In(B_k)$.\\

Par indépendance de données, $\forall i \neq k, In(B_k) \cap Out(B_i) = \emptyset$.\\
Donc $s$ ne dépend que de $B_k$, i.e. $s$ ne dépend pas de l'ordre d'exécution.\\

$B$ est donc parallélisable.
\end{proof}

\newpage

\begin{theorem}
Les deux boucles imbriquées d'indices $i$ et $j$ sont permutables si et seulement si il n'existe pas de dépendances de données ou de sorties entre des instances $(i+k_i, j-k_j)$ ou $(i-k_i, j+k_j)$ et l'instance $(i,j)$, $(k_i, k_j) \in \mathbb{N}^*$
\end{theorem}

\begin{proof}[Preuve]
La permutation des boucles ne va changer que partiellement l'ordre d'exécution.\\
Pour chaque instance $(i,j)$, on note:
\begin{itemize}
\item $After_{1}(i,j)$ l'ensemble des instructions après celle d'indice $(i,j)$ lorsque $j$ est à l'interieur : ce sont les instances $(i,j+k_j)$ ou $(i+k_i,j+l_j)$ avec $k_i,k_j \in \mathbb{N}^*$ et $l_j \in \mathbb{Z}$
\item $After_{2}(i,j)$ l'ensemble des instructions après celle d'indice $(i,j)$ lorsque $i$ est à l'interieur : ce sont les instances $(i+l_i,j+k_j)$ ou $(i+k_i,j)$ avec $k_i,k_j \in \mathbb{N}^*$ et $l_i \in \mathbb{Z}$
\item $Before_{1}(i,j)$ l'ensemble des instructions après celle d'indice $(i,j)$ lorsque $j$ est à l'interieur : ce sont les instances $(i,j-k_j)$ ou $(i-k_i,j+l_j)$ avec $k_i,k_j \in \mathbb{N}^*$ et $l_j \in \mathbb{Z}$
\item $Before_{2}(i,j)$ l'ensemble des instructions après celle d'indice $(i,j)$ lorsque $i$ est à l'interieur : ce sont les instances $(i+l_i,j-k_j)$ ou $(i-k_i,j)$ avec $k_i,k_j \in \mathbb{N}^*$ et $l_i \in \mathbb{Z}$
\end{itemize}

On peut permuter si et seulement si, pour chaque instruction $(i,j)$, il n'y a aucune dépendance de données ou de sorties entre $(i,j)$ et les autres instructions qu'on permute.\\

Ces instructions sont $After_{1}(i,j) \cap Before_{2}(i,j)$ et $After_{2}(i,j) \cap Before_{1}(i,j)$.\\

Soit l'instance $(p,q) \in After_{1}(i,j) \cap Before_{2}(i,j)$.

\[
\left\{
    \begin{array}{ll}
        (p,q) &= (i,j+k_j) \mbox{ ou } (i+k_i,j+l_j), (k_i,k_j) \in \mathbb{N}^*$ et $l_j \in \mathbb{Z}\\
        (p,q) &= (i+l_i,j-k_j) \mbox{ ou } (i-k_i,j), (k_i,k_j) \in \mathbb{N}^*$ et $l_i \in \mathbb{Z}
    \end{array}
\right.
\mathrm{soit} ~~
\left\{
    \begin{array}{l}
        p = i+k_i  \\
        q = j-k_j
    \end{array}
\right.
\]

De la même façon, $After_{2}(i,j) \cap Before_{1}(i,j)$ donne $(p,q) = (i-k_i, j+k_j)$.\\

Donc les boucles sont permutables si et seulement si, $\forall (i,j), \forall (k_i, k_j) \in \mathbb{N}^*$, il n'y a aucune  indépendance de données entre les instances $(i,j)$ et $(i-k_i, j+k_j)$ ou $(i+k_i, j-k_j)$.

\end{proof}

\end{document}