\documentclass{beamer}
%
% Choose how your presentation looks.
%
% For more themes, color themes and font themes, see:
% http://deic.uab.es/~iblanes/beamer_gallery/index_by_theme.html
%
\mode<presentation>
{
  \usetheme{metropolis}      % or try Darmstadt, Madrid, Warsaw, ...
  \usecolortheme{default} % or try albatross, beaver, crane, ...
  \usefonttheme{default}  % or try serif, structurebold, ...
  \setbeamertemplate{navigation symbols}{}
  \setbeamertemplate{caption}[numbered]
}

\usepackage[frenchb]{babel}
\usepackage[utf8x]{inputenc}
\usepackage[T1]{fontenc}
\usepackage{extramarks} % Required for headers and footers
\usepackage{listings} % Required for insertion of code
\usepackage{amsfonts}
\usepackage{amsmath}
\usepackage{hyperref}
\usepackage{textcomp}
\usepackage{stmaryrd}
\usepackage{framed}

\lstset{texcl=true, columns=flexible,basicstyle=\ttfamily}

\title[Parallélisation de l'algorithme de Jacobi]{Projet : Parallélisation de la Méthode de Jacobi pour la résolution de systèmes linéaires}
\author{Plessia Stanislas}
\institute{}
\date{Mars 2018}

\begin{document}

\begin{frame}
  \titlepage
\end{frame}

\section{Méthode de Jacobi}

\subsection{Définition du problème}

\begin{frame}{Définition du problème}
  Soit :
  \begin{itemize}
   \item $n \in \mathbb{N}$ 
   \item $A = (a_{i,j})_{i,j \in \llbracket 1,n \rrbracket^2}$ matrice carrée de taille $n$
   \item $b = (b_i)_{i \in \llbracket 1,n \rrbracket}$ vecteur de taille $n$.
  \end{itemize}

  On cherche alors le vecteur $x = (x_i)_{i \in \llbracket 1,n \rrbracket}$ tel que :

  \begin{equation} 
      \label{eq:probleme}
      Ax = b
  \end{equation}
\end{frame}

\subsection{Résolution}

\begin{frame}{Résolution: Decomposition de $A$}
  On décompose $A$ en deux matrices :
  \[
  A =
    \begin{pmatrix}
    a_{1,1} & 0 & \cdots & 0 \\
    0 & a_{2,2} & \cdots & 0 \\
    \vdots  & \vdots  & \ddots & \vdots  \\
    0 & 0 & \cdots & a_{n,n}
   \end{pmatrix}
   +
    \begin{pmatrix}
    0 & a_{1,2} & \cdots & a_{1,n} \\
    a_{2,1} & 0 & \cdots & a_{2,n} \\
    \vdots  & \vdots  & \ddots & \vdots  \\
    a_{n,1} & a_{n,2} & \cdots & 0
   \end{pmatrix}
  \]

  Notons $D$ la matrice diagonale et $R$ le reste.
  Comme on suppose les coefficiens diagonaux non nuls, $D$ est trivialement inversible et on peut transformer l'équation \eqref{eq:probleme} :
  \[
    x = D^{-1}(b - Rx)
  \]
\end{frame}

\begin{frame}{Résolution: Solution Itérative}

  Un solution itérative peut \^etre construite de la manière suivante :
  \[
      \left\{
      \begin{array}{l}
          x^{(0)} = \vec{0}\\
          x^{(k+1)} = D^{-1}( b - Rx^{(k)})
      \end{array}
      \right.
  \]
  D'après le théorème du point fixe, on peut montrer qu'on a alors :
  \[
  \lim_{k \to \infty}x^{(k)} = x \Leftrightarrow \rho(-D^{-1}R) < 1
  \]
  On obtient alors la formule de récurrence sur les coefficients :
  \begin{align}
      \forall k \in \mathbb{N}, \forall i \in \llbracket 1,n \rrbracket, 
      \qquad x_{i}^{(k+1)} = \dfrac{1}{a_{i,i}}(b_i - \sum_{i \neq j}a_{i,j} x^{(k)}_j)
  \end{align}

\end{frame}

\begin{frame}{Résolution: Condition suffisante de convergence}
  Une condition suffisante pour assurer la convergence de la méthode de Jacobi est la suivante:\\

  Soit $\lambda$ valeur propre de $C$ et $y_{\lambda}$le vecteur propre associé, on a :
  \begin{align*}
    |\lambda|\cdot||y_{\lambda}||_{\infty} &= |\dfrac{1}{a_{i,i}}| \cdot |\sum_{j \neq i}a_{i,j}y_{\lambda{j}}|\\
    &\leq |\dfrac{1}{a_{i,i}}| \cdot \sum_{j \neq i}|a_{i,j}| \cdot ||y_\lambda||_{\infty}
  \end{align*} 

  Donc on a $\forall (i,j) \in \llbracket 1,n \rrbracket^2 \quad |a_{i,i}| > \sum_{i \neq j}|a_{i,j}| \implies \rho(C) < 1$,\\
  ie $C$ est a diagonale strictement dominante.
\end{frame}

\section{Implémentation de la Méthode}

\begin{frame}
\end{frame}

\end{document}